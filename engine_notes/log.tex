\documentclass{article}

\usepackage{listings}
\usepackage{xcolor}

\definecolor{dkgreen}{rgb}{0,0.6,0}
\definecolor{lightgray}{rgb}{0.975,0.975,0.975}
\lstdefinelanguage{cpp}{
      backgroundcolor=\color{lightgray},  
      basicstyle=\footnotesize \ttfamily \color{black} \bfseries,   
      breakatwhitespace=false,       
      breaklines=true,               
      captionpos=b,                   
      commentstyle=\color{dkgreen},   
      deletekeywords={...},          
      escapeinside={\%*}{*)},                  
      frame=lines,                  
      language=C++,                
      keywordstyle=\color{purple},  
      morekeywords={BRIEFDescriptorConfig,string,TiXmlNode,DetectorDescriptorConfigContainer,istringstream,cerr,exit}, 
      identifierstyle=\color{black},
      stringstyle=\color{blue},      
      numbers=left,                 
      numbersep=5pt,                  
      numberstyle=\tiny\color{black}, 
      rulecolor=\color{black},        
      showspaces=false,               
      showstringspaces=false,        
      showtabs=false,                
      stepnumber=1,                   
      tabsize=4,                     
      title=\lstname,                 
}


\begin{document}

\subsection{Log}
The log module consists of several parts:

\begin{itemize}

    \item Logger\\
    The logger is the top-level class of the code flow, containing instances of all the other log classes. 
	In particular it contains an output manager.

    \item Message\\
    A message is as it says on the tin, a message. 
	The class contains the message itself, the loggers name, and a level. 
	The level is the type of message it is, which can be either info, verbose, warning, error, fatal or debug.

    \item Output\\
    Output is the abstract base class which defines how the messages are displayed, and in which manner they are displayed. 
	You can define a output format which allows several built in variables to display certain information, like the logging level and time stamp. 
	You use this class my making a child class, and then overriding the outputMessage class to display the messages. 
	One could for example make a child class that writes the logs to the console, or saves the logs to file.

    \item OutputManager\\
    Handles all the different variations of outputs. 
	A pointer of this class is held in most of the other classes, and is called upon to do the logging to all the different log outputs.

    \item OutputStream\\
    A child implementation of the Output base class. 
	This class makes it possible to output the logs via ostream, aka cout. 
	An instance of this class is held in the output manager.

\end{itemize}

The Logger class has instances all over the NOX-Engines code base. This allows the logging functionality to be utilized everywhere it's needed. 
A particular logger will receive a string and convert it to a proper message instance, and send it to the central hub of the logging system, the OutputManager.
The OutputManager will then take that message and then send it to each instance of Output the manager has. Each output will then format, change or transform the message however it pleases, for example by displaying it in the console.
\end{document}