\subsection{Entity Lifecycle}
The entity life cycle follows the original nox lifecycle,
however to clear up some misunderstandings this document contains a deeper explanation of how the lifecycle will function within the new ECS.
Below are the following states than a "component" can be in, and how it transitions between states.\\
All the transitions are async, i.e. a component that is requested to be deactivate is not deactivated until the end of the current frame.
However going through more than one state still only happens in one "step".
i.e. Going from raw memory-to-active is done in one frame, not three, same with active-to-raw memory.

\ptparagraph{Raw Memory}
Just like everything else, a component starts out as raw memory. 
The reason we include this state is to clear up any confusion with the existing nox engine's destroy transition. 
The transition from Raw Memory to a hibernation state is done through the constructor of the object. The creation of the component is then done through the assingComponent function in the EntityManager.

There is an optional step between the constructor and going into hibernation, and that is initialization, which can be requested when assigning a component.

Raw memory is not really a state in the life cycle per say, it is only here to clear up any confusion in relation to destroy within the existing NOX Actor system. 
Raw memory is before a the operator new or after operator delete has been run on an object.
The reason for this inclusion is that the new ECS will be pooling and re-using a lot of its memory.

\ptparagraph{Hibernating}
In the hibernating state, the object only holds its own variables in place, and is not part of the update loop.

The hibernation stage is supposed to be for long term inactivity.
It is sort of like after an onStop has been run within the Android Lifecycle. i.e. The object can still be rebuilt, but it has let go of most of its resources.

Going from hibernation to inactive is done through the Awaken transition.\\
Going from hibernation to raw memory is done calling the removeComponent or removeEntity function from the EntityManager, 
which will run the destructor of the component.

\ptparagraph{Inactive}
The inactive state is for short term inactivity, most resources should still be kept within the component, 
as reactivating it should be a lightweight transition.
The component's update function is still not part of the overall update loop.

Going from Inactive to Active is done with the Activate transition.\\
Going from Inactive to hibernation is done with the Hibernate transition

\ptparagraph{Active}
When the component is active, its update function is run as normal.

Going from active to inactive is done with the deactivate transition.

Whether or not a component shall receive events outside of its update functions is decided when the component is registered.
