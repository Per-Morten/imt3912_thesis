\subsection{Update Cycle}
The update cycle goes through several different steps. The following section elaborates a bit on those steps. The steps are executed in the following sequence.
\begin{enumerate}
    \item Distribute Logic Events
    \item Update
    \item Distribute Component Events
    \item Deactivate Requested Components
    \item Hibernate Requested Components
    \item Remove Requested Components
    \item Create Requested Components
    \item Awake Requested Components
    \item Activate Requested Components
\end{enumerate}

The user is allowed to call each of these functions in the desired order, 
or call a "handle all" type function which will go through all of the steps in the sequence above.

\ptparagraph{Distribute of Logic Events}
The distribution of logic events is the first step that will be run.
In this step the EntityManager will loop through each component and distribute logic events coming from other parts of the NOX Engine. 
Components are allowed to respond to these events by broadcasting new ones, 
and this step will not finish until there are no more events.
This means that it is also the users responsibility to ensure that they don't
end up in a event cycle.

\ptparagraph{Update}
After the distribution of logic events is finished the update step will start. In this step, the EntityManager will loop through each component,
and call the update function specified by that component type's MetaInformation.
The order in which components are updated is unspecified. So the users cannot rely on any ordering of the update functions.

\ptparagraph{Distribute Component Events}
After the update loop has been run the EntityManager starts distributing the different component events that has been queued during the update function.

\ptparagraph{Deactivate Requested Components}
If any components are requested to be deactivated, the EntityManager will go through them and call their respective deactivate function.

\ptparagraph{Hibernate Requested Components}
If any components are requested to go into hibernation state, the EntityManager will go through them and call their respective hibernate function.

\ptparagraph{Remove Requested Components}
Any components that is requested to be removed will have their destructor called in this phase.

\ptparagraph{Create Requested Components}
Any creation of components will happen during this phase. This includes both running constructors, and running initialization if that is requested.
One of the reason for doing this after having removed components is so we can potentially reuse some of the memory that was just freed.

\ptparagraph{Awake Requested Components}
Any components that has been requested to awake will have their awake function called in this phase.

\ptparagraph{Activate Requested Components}
Any Components that has been requested to be activated will have their activate function called in this phase.
