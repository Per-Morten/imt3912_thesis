\section{Multi-threading}
\subsection{Update}
The main part of the ECS that is going to be multi-threaded is the EntityManager step functions.
The strategy of the threading implementation will be a fork-and-join\cite{wikipedia_fork_and_join}, 
where the different Operations that are run on components will be run in parallel based on their DataAccess value.

\subsection{User Responsibilities}
The new ECS is as much as possible an opt-out system. By default the system should always assume "the worst" and go for the least performing but safest methods. 
As soon as the user starts opting out of these safety features, 
it is their responsibility to ensure that they are using the system correctly, and are following the guidelines of the documentation.

\subsection{Dependencies}
While the DataAccess values INDEPENDENT and READ\_MODIFY\_WRITE has quite obvious implementations of the threading model, 
the same cannot be said for the READ\_COMPONENT value. 
The system has to ensure that a component never sees a torn read, i.e. an object that is halfway through its update function.
There are some ways of solving this.

\ptparagraph{Direct Acyclic Graph}
One way could be to try and order the functions based on their listed dependencies, and create a DAG. 
However problems will arise in the case of cyclic dependencies within the components dependencies. 
One way of solving that could be to try and mark some of the dependent component as READ\_MODIFY\_WRITE, and continue trying to create a DAG. I.e. Try to remove the components that have cyclic dependencies.

\ptparagraph{State Caching}
Another perhaps easier way is to cache the current state of the components that are dependencies, in this case, when a Operation does a read, it would read from the cached representation of a component, while the proper component is getting updated.

However, the exact solution to this will be decided on when we start implementing.
