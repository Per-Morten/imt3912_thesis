\subsection{DataAccess}
Interface: Figure~\ref{lst:data_access}\\\noindent
The data access enum is used to identify what sort of data access an operation does with an outside system,
identifying which operations that can be run in parallel and which ones cannot.
It contains three values that give off different indications.
If another value than READ\_MODIFY\_WRITE is given, it is the users responsibility to ensure that that sort of DataAccess is followed.

\textbf{Note:} The DataAccess only regards interaction with external non thread-safe systems.
If an operation only interacts with systems that have some thread protection, it can be labeled as INDEPENDENT.

\ptparagraph{INDEPENDENT}
INDEPENDENT means that the operation is entirely thread safe, it does not interact with any non thread-safe functionality outside of its own modification.

\ptparagraph{READ\_COMPONENT}
This operation will involve reading from another component, that is not thread-safe.

\textbf{Note:} In this case, the TypeIdentifiers of the components that shall be read must be listed, so the ECS can figure out which sequence to run the operations in.

\ptparagraph{READ\_MODIFY\_WRITE}
This value signifies a "hands-off" approach within the ECS,
operations marked with READ\_MODIFY\_WRITE will always run in a single-threaded environment.

\textbf{Note:} To avoid accidentally introduced race conditions, this is the assumed behavior unless another DataAccess value is specified.
