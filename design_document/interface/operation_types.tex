\subsection{OperationTypes}
Interface: Figure~\ref{lst:operation_types}\\\noindent
"Replacing" virtual functions are OperationTypes.
OperationTypes are just function pointers, that accepts Components as arguments.
Using function pointers instead of virtual functions allows us to loose
the vtable variable that is silently introduced into each instance,
it should also help with cache usage, as we always do the same operation on a range of types.
It also allows us to do certain "opt-in" type safety features,
like wrapping lambdas around operations and casting to the correct type before calling specified functions.

\ptparagraph{RangedOperation}
RangedOperations are given a range of objects to work on, "first" and "last".
These are functions identical to the iterator based approach of the STL library.\\
The "first" parameter points to the first element in the range, while the "last" parameter points to a past-the-end element of the range.\\
It is the responsibility of the function implementor to cast these pointers to the correct type, 
to ensure the correct step size when incrementing the pointers.

\ptparagraph{SingularOperation}
SingularOperations are simple pointers to functions that only take an IComponent* parameter.\\
As with the RangedOperation, it is the function implementors responsibility to cast the parameter to the correct type.

\ptparagraph{LogicEventOperation}
LogicEventOperations are used when components must respond to types coming from the rest of the engine.\\
Since interest of components are often registered on a component type basis,
this function will be applied to the entire range of a component type.

\ptparagraph{ComponentEventOperation}
The ComponentEventOperation is a function that is called on inter component communication, i.e, 
sibling or non sibling components send messages to each other.

\ptparagraph{InitializationOperation}
The initializationOperation is used for initializing components with a json value.
