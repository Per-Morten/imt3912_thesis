\subsection{OperationTypes}
\lstinputlisting[language=cpp, caption=OperationTypes.cpp]{../code_samples/operation_types.cpp}
"Replacing" virtual functions are OperationTypes.
OperationTypes are just function pointers, that accepts IComponents as arguments.
There are two operation types, RangedOperation and SingleOperation.

\ptparagraph{RangedOperation}
RangedOperations are given a range of objects to work on, "first" and "last".
This is functions identical to the iterator based approach of the STL library.\\
The "first" parameter points to the first element in the range, while the "last" parameter points to a past-the-end element of the range.\\
It is the responsibility of the function implementor to cast these pointers to the correct type, 
to ensure the correct step size when incrementing the pointers.

\ptparagraph{SingularOperation}
SingularOperations are simple pointers to functions that only take an IComponent* parameter.\\
As with the RangedOperation, it is the function implementors responsibility to cast the parameter to the correct type.

\ptparagraph{LogicEventOperation}
LogicEventOperations are used when components must respond to types coming from the rest of the engine.\\
Since interest of components are often registered on a component type basis,
this function will be applied to the entire range of a component type.

\ptparagraph{ActorEventOperation}
The ActorEventOperation is a function that is called on inter component communication, i.e, 
sibling or non sibling components send messages to each other.
