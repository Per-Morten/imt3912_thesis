\subsection{SmartHandle}
Interface: Figure~\ref{lst:smart_handle}\\\noindent
A smart handle is a simple wrapper for an non owning pointer to a component, with the dereferencing semantics of std::unique\_ptr. 
A smart handle allows for simple "caching" of a component reference.
The smart handle is created with the pointer to an object, 
and the "generation" when the reference was captured. 
When a handle is dereferenced the generation of the handle is checked against the current generation of the components MetaInformation. 
If the current generation is newer than the one contained within the handle, 
the handle will query the EntityManager for an up-to-date reference.
If an up-to-date reference can't be found, 
the handle will return a nullptr, meaning that the component is destroyed.
The concept of generations was inspired by Reinalter\cite{molecular_musings_internal_references}.

A generation increment happens every time an operation is done on a collection of components that may invalidate a pointer into the collection.
For example when reallocating a collection of components to a new location,
or when sorting a collection of components.
However, "out-of-thin-air" generation increments shall not happen, the behavior of when a generation is incremented is deterministic.
