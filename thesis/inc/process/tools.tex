\section{Tools}
\subsection{Git}
Version control was done through git. Git is something we were both familiar with, and is also the version control
system that Suttung is using. However, while Suttung uses gitlab for storing their repositories we originally
chose to use bitbucket. We hopped over to github as we were more interested in their tools for issue tracking,
and development logging.

\subsection{Project Management}
Different project management platforms were used during the development of the project.
Originally we used Atlassian's Jira platform, which we perceived as being to much overhead for a project
consisting only of two people. We then moved over to simply using bitbuckets issue tracking system, before finally
settling on github. Github allowed us both access to a proper issue tracker, a project board, as well as good tools
for doing code review on pull requests.

\subsection{Development Environment}
Originally we planned to implement the NOX ECS on Windows, using Microsoft Visual Studio as our coding environment.
However, after seeing some of the tools the Linux platforms had to offer especially in terms of code profiling,
we decided to move over to Ubuntu.
We stayed on Ubuntu for the whole project, coding in Sublime Text, and building with the CMake build tools.
Windows was used from time to time to solve issues which required more sophisticated debugging tools.

\subsection{Fault Detection}
There were several times during the development of this project that we would run into bugs like memory corruption etc,
especially when implementing the component collection.
In these cases Valgrind's memcheck was used to ensure program correctness, and was a helpful tool in the debugging process.
