\section{Templates}
\label{sec:requirements_templates}
\todo{Ask Mariusz how much is actually needed here}
C++ has a infamous for it's long compile times, the reason for the long compile times is influenced by several factors.
\cite{stack_overflow_why_does_cpp_compilation_take_so_long}
Some of these factors include:
\begin{itemize}
    \item
    Textual inclusion of header files

    \item
    Parsing of complex syntax

    \item
    Linking

    \item
    Templates

    \item
    Template Meta Programming
\end{itemize}

Templates is a language feature within C++ that allows programmers to write generic type-safe code.
When compiling a template class in C++, the compiler will generate code for each instantiation of
a template class.
However this process of generating code for template instantiations only happens on a per translation unit basis.
This means that the compiler will have to compile the same template instantiation of a class multiple times,
these duplicate template instantiations are then removed during the linking phase.
In addition to this, templates have to be written within header files,
meaning that any change to that header file will lead to a recompilation of all files that includes that header file.
\cite{dr_dobbs_cpp_compilation_speed}
\todo{Could perhaps here mention how template meta programming also can lead to some positive things, with Vittorio Romeo's library as an example}

Suttung expressed a wish avoid a heavy compile time increase, as rapid iteration is hampered by having to wait for a program to compile.
We therefore decided together with Suttung to try and avoid heavy templated solutions to keep acceptable build times.
