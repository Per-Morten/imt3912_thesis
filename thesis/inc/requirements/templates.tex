\section{Templates}
\label{sec:requirements_templates}
\todo{Ask Mariusz how much is actually needed here}
C++ is infamous for its long compile times, caused by several factors
\cite{stack_overflow_why_does_cpp_compilation_take_so_long}.
Some of these factors include:
\begin{itemize}
    \item
    Textual inclusion of header files

    \item
    Parsing of complex syntax

    \item
    Linking

    \item
    Templates

    \item
    Template Meta Programming
\end{itemize}

Templates is a language feature within C++ that allows programmers to write generic type-safe code.
When compiling a template class in C++, the compiler will generate code for each instantiation of a template class.
However, this process of generating code for template instantiations only happens on a per translation unit basis\footnote{Translation unit is basically a C++ source file}.
It means that the compiler will have to compile the same template instantiation of a class multiple times.
These duplicate template instantiations are removed during the linking phase.
In addition to this, templates have to be written within header files,
meaning that any change to that header file will lead to a recompilation of all files that includes that header file\cite{dr_dobbs_cpp_compilation_speed}.

Suttung expressed a wish to avoid a significant compile time increase, as waiting for the program to compile hinders a rapid iteration process.
Therefore, we decided together with Suttung to try to avoid heavy templated solutions to keep acceptable build times.

\requirement{NOX ECS shall avoid heavy template usage.}{req:templates}
\reqcomment{This is done to avoid further increase in compile times}
