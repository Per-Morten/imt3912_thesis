
\section{Performance}
Performance is an important aspect of the video game industry.
As a result of this, Suttung and the authors decided upon certain restrictions and goals.

\subsection{Virtual Functions}
\label{subsec:requirements_performance_virtual_functions}
Virtual functions are C++'s way of supporting dynamic function binding, allowing for run-time polymorphism.
With the help of virtual functions, objects derived from a base class can override base class functionality, even when the derived type is unknown at compile time.
The C++ ISO Standard does not mandate the implementation of virtual functions.
However, many compilers choose the same tactic; creating a virtual table\cite[p. 140]{game_engine_architecture}

The virtual table can be viewed as a data structure containing function pointers to virtual functions.
Each instance of a derived object is given a hidden member variable, which is a pointer to their class's respective virtual table.
When invoking a virtual function, the virtual table pointer is used to look up which function to execute.
The virtual table pointer can be placed anywhere, but is usually either the first or last member of an object.
Multiple inheritance is often implemented by adding more virtual tables and virtual table pointers\cite{codersource_virtual_functions}\cite[31:12]{andrei_alexandrescu_quick_code_quickly}.

The introduction of a hidden member variable adds a certain space overhead for each instance of a polymorphic class.
A few factors influence the exact space overhead of the implementation.
\begin{itemize}
    \item
    The size of a pointer within the given architecture.

    \item
    The number of base classes.

    \item
    Total alignment requirement of the class.
\end{itemize}

Invoking a virtual function also adds some overhead regarding extra lookups, the severity varies between architectures,
as well as from situation to situation.
The overhead can, for example, be influenced by whether or not the function is in the CPU's instruction cache, or whether or not the CPU has made a branch miss prediction, which can lead to pipeline flushes and stalls\cite{scott_meyers_cpu_caches_and_why_you_care}\cite[Data Locality]{game_programming_patterns}.
As a result of invoking virtual functions at runtime, the compiler might only see the invocation as an opaque function call,
possibly disabling certain optimizations, like inlining, reordering, compile time calculations, or even removing empty function calls.
These tradeoffs are known, and compilers can to some extent optimize these issues.
For example, if a type is known at compile time, then a compiler might do de-virtualization,
meaning that it will not do a lookup on a virtual pointer table, but rather do a direct function call\cite{lazarenko_devirtualization}.

There is rarely a need for direct run-time polymorphism in the NOX Engine, especially within the actor system.
In the original NOX Actor system, all the different components were known at compile time.
As a result of this Suttung and the authors decided to try to find another mechanism for handling "polymorphic" behavior, leading to the following requirement.

\requirement{NOX ECS shall provide functionality akin to virtual, for polymorphic types known at compile time.}{req:virtuals}
\reqcomment{This functionality should ideally be provided in an efficient manner}

\subsection{Contiguous Memory}
\label{subsec:requirements_performance_contiguous_memory}
Modern day CPU architecture has a different set of problems concerning optimization than in earlier history.
Much focus has gone into making the CPU execute instructions faster, through parallelization, pipelining, and so forth.
Unfortunately, memory access efficiency has not managed to keep up with the raw speed of the CPU.
Caching and prefetching mechanisms are introduced to compensate for the inefficient memory access\cite[p. 153]{game_engine_architecture}.
Taking advantage of these caching mechanisms can lead to great boosts in performance, which is desirable for Suttung.
However, these mechanisms do require that, for example, we take memory layout into consideration, with a preference
to contiguous memory.

\requirement{NOX ECS should have a preference towards contiguous memory schemes, to make better use of modern caching mechanisms}{req:contiguous_memory}

\subsection{Multi-threading}
\label{subsec:requirements_performance_multi_threading}
To further improve performance Suttung has requested that the new ECS should be multi-threaded, allowing them to utilize multi-core hardware more efficiently.
The multi-threading should preferably be something that happens in the background so that the programmers do not have to worry much about synchronization.

\requirement{NOX ECS shall supply solutions for writing code that can be parallelized.}{req:multi_threading}
\reqcomment{This should not require large investments by the programmer}
