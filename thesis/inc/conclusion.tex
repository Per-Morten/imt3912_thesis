\chapter{Conclusion}
\label{chap:conclusion}
The thesis has shown and discussed our implementation of the multi-threaded entity
component system; NOX ECS, within the NOX Engine. NOX ECS is an alternative to the NOX Actor system.
We have performed benchmark tests with the NOX Actor system as reference,
and benchmarks of different internal optimizations with our normal NOX ECS behavior as reference.

We found clear performance differences between the architecture of NOX ECS and the NOX Actor system,
even though the tests were flawed to a certain extent, as discussed.
The internal optimization benchmark tests showed no such clear performance difference.
Based on these two findings, we argue that our main conclusion is that the architecture is more important than enhancing optimization.
We recommend that a systems architecture should be based on the expected usage patterns of the system,
and that optimization should be evaluated from an architectural perspective, as well as a local perspective.

These conclusions and findings can be relevant for other parties interested
in entity component systems, performance and multi-threading.
We believe that once finished the NOX ECS will be a valuable alternative to the NOX Actor system.

\paragraph{Learning Outcome}
We are extremely thankful for the experiences gained during the project.
Developing the NOX ECS has allowed us to learn more about optimization techniques,
as well as a deeper understanding of how the C++ language works.
We have also gained experience with integration and development within a larger codebase,
and the challenges that this incur.
These experiences are great contributions to our long journey to becoming better developers.
