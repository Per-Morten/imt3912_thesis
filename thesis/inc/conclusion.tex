\chapter{Conclusion}
\label{chap:conclusion}
The thesis has shown and discussed our implementation of the multi-threaded entity
component system; NOX ECS, within the NOX Engine.
NOX ECS is an alternative to the NOX Actor system.
We have performed benchmark tests with the NOX Actor system as reference,
and benchmarks of different internal optimizations with our normal NOX ECS behavior as reference.
We argue that the differences between NOX ECS and the NOX Actor system's results were clear enough for us
start a discussion on the performance differences of the two.
This was done even thought the tests were flawed to a certain extent.
The lack of a clear performance enhancing optimization within the internal benchmarks further
lead us down towards our conclusion on the importance of architecture.
We argued that a systems architecture should be based on the expected usage patterns of a system,
and that optimization should be evaluated from an architectural perspective, as well as a local one.
These conclusions and findings can be relevant for other parties interested
in entity component systems, performance and multi-threading.
We believe that once finished the NOX ECS will be a valuable addition to the NOX Engine.


\paragraph{Learning Outcome}
We are extremely thankful for the experiences gained during the project.
Developing the NOX ECS has allowed us to learn more about optimization techniques,
as well as a deeper understanding of how the C++ language works.
We have also gained experience with integration and development within a larger codebase,
and the challenges that this incur.
These experiences are great contributions to our long journey to becoming better developers,


Skriv ut litt mer om konklusjonen, få ut essensen av

Vurderingen vår er at konklusjonen med NOX ECS og Actor er ganske god, fordi
forskjellene er såppass store. Dette er selv om det er noen svakheter med dataen som denne konklusjonen er tatt av.

The main conclusion reached from these findings is that choices taken on implementations
should be based on the expected usage patterns of the system, the flexibility required,
and the potential costs of the different implementations.
Additionally optimizations should also be seen from an architectural standpoint rather than
just a local standpoint. As shown in the tests, the main contributing factor to NOX ECS's performance
is the architectural design, rather than one individual optimization.


We argue that the flaws with our tests were taken into account when during discussion.

Even though our tests and findings were not without flaws, we believe
that the conclusion of thinking of performance from an architectural standpoint based on
expected usage patterns can help improve a programs efficiency.
