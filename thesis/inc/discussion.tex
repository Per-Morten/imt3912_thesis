\chapter{Discussion}
\label{chap:discussion}

\section{Findings}
While the benchmarks and implementations were created with the NOX Engine in mind,
we believe these findings can also be generalized and related to other systems.

\subsection{Layered Execution}
As shown in the benchmarks the layered execution model can provide a significant performance boost,
and is a nice way to introduce multi-threading into a system relatively invisible.
However, there are usages patterns which is non optimal for the layered execution model.
These are mainly cases where there is not enough processing requirements to justify the synchronization costs.
For example if the user is sending a lot of direct events, rather than broadcasted ones,
and the processing required for receiving events is quite short.

\subsection{Virtual Replacement}
Having an own virtual function replacement might remove some space overhead, and might perform better in some cases,
however that also requires all the types of polymorphism to be known at compile time, which might
not be possible.
The main advantage of using the meta information structure we proposed is that we are able to remove empty
virtual calls.
In the cases where we were not removing unused functions there did not seem to be that much improvement,
as in all the test cases the component types were sorted by type, leading to more instruction friendly cache
traversal.

\subsection{Memory Reuse}
Being able to reuse memory did not provide as big improvement as we had originally envisioned.
However, we are unsure if this would have lead to better results on mobile devices.
Gregory\cite[p.262]{game_engine_architecture} that modern PC's have more advanced CPU's and virtual
memory, meaning that dynamic memory allocation is a lot cheaper than on devices with more limited hardware.

\subsection{Lock Free}
Based on what we have seen in the test cases, one should be careful of writing lock free algorithms.
In many cases a naive implementation with a simple lock can perform as well as a complex lock free algorithm.
While we were not able to test our code on a weakly ordered platform we did not see any great improvement
of the lock free structures over the naive locked ones, and the lock free algorithms had a very limited interface.

\subsection{Separate Data for Separate Threads}
Some of the performance issues within the NOX ECS is a result of synchronization overhead,
and could potentially have been solved by separating the data on a per thread basis.
We fell into the trap of thinking to much on data sharing, which did hurt the NOX ECS performance.
In general we believe that if feasible one should avoid sharing data between different threads.
For example each thread could have gotten its own allocator, which would remove some synchronization overhead.

\subsection{Templates}
While templates often increase build times, it is clear that the extra help they provide in
fault checking at compile time, and possible optimizations that are allowed through type traits
are extremely valuable.
However, as shown in this thesis it is possible to write reusable code without the need for templates,
and without having to sacrifice good memory usage, which can be harder to do when just working with
base classes and interfaces.

\subsection{String identifiers}
String identifiers are very flexible, however, this flexibility does come a significant cost, as seen
in the numerous unique components internal test.
It is important to make a decision on whether or not this flexibility is really utilized, and worth
the cost that it brings.

\subsection{Entities and Components}
The separation of Actors holding components to entities identified by components did come with some
desirable benefits. The clearer separation allows users to avoid paying for functionality that they
do not use. It also made a clearer distinction of how data was touched, which allowed for easier efficient
multi-threading integration.
However, the decision to go with an ECS should be based on the expected usage patterns, and the
need for communication between the components.
A lot of direct communication will need a better system then what was implemented in the NOX ECS.

\subsection{Cache Friendly Traversal}
As seen from the numerous unique components internal test there is a performance gain from
ensuring more cache friendly traversals of memory.
If possible high frequently visited memory should be stored relatively contiguously,
and operated on in such a manner.

\subsection{Unnecessary Functionality}
The NOX Actor system offers a lot of functionality, however, this functionality is not necessarily always used,
however the user is still paying for the overhead of including this functionality, like the children components
stored in the NOX Actor system.
We believe that implementations should stay decoupled by offering no more functionality than what is needed
and requested by the user, and that we were able to gain benefits from this line of thought.
\todo{Think about if this should really be here!}

\subsection{Conclusion}
The main conclusion reached from these findings is that choices taken on implementations
should be based on the expected usage patterns of the system, the flexibility required,
and the potential costs of the different implementations.
Additionally optimizations should also be seen from an architectural standpoint rather than
just a local ones. As shown from the tests, it seems to be the main architectural design
of the NOX ECS that contributes to its performance, rather than one individual optimization.
\todo{Check this with Simon}

\subimport{discussion/}{criticism.tex}
