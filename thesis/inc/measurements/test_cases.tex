\section{Test Cases}

\subsection{Purpose}
The tests purpose is to measure the quality of the different optimizations and choices made within the ECS.
For each test the results are supposed to give an indication to the performance differences between NOX ECS and the NOX Actor system.
Additionally, internal tests are used to compare the performance of the different optimizations within the NOX ECS implementation.

\subsection{Compiler}
We only compiled the test cases with GCC, instead of the original plan of also included Clang and MSVC.
This decision was motivated by the time constraints, after discovering how long the tests took.

\subsection{Platforms}
The original plan was to run the tests on both Windows and Linux. With additional time,
tests would also be run on a mobile platform\reqref{req:mobile_platform}.
Due to time constraints, the tests were only run on Linux.

\subsection{Testing Platform}
\subsubsection{Hardware}
All test cases were run on a two year old "Lenovo Y70-70 Touch 80DU", a selection of the different hardware components are listed in
table \ref{tab:hardware_info}. A more detailed version can be seen in appendix \todo{Insert appendix ref}.

\begin{table}
\begin{center}
\begin{tabular}[label=tab:hardware_info]{ l | r }
    Components & Info \\
    \hline
    CPU & Intel(R) Core(TM) i7-4710HQ CPU @ 2.50GHz \\
    Cores & 4 \\
    Threads & 8 \\
    Architecture & x86\_64 \\
    System Memory & \SI{16}{\giga\byte} \\
\end{tabular}
\caption{Hardware info}
\label{tab:hardware_info}
\end{center}
\end{table}

\subsubsection{Precautions}
Measuring performance on modern architectures is quite difficult. The complexity of modern operating systems and hardware means
it is harder for us to take control to ensure that our tests are run in isolation.
Operating systems can context switch out the test, to run other tasks that needs to be done.
The hardware can apply cpu frequency scaling, to avoid wasting unnecessary power.
We were not able to turn off these features. However, we tried to mitigate their impact.

\paragraph{Program Isolation}
When performing tests the computer would not have any other user programs running in the background.
Additionally wifi was turned off, and all IO devices were disconnect from the computer.
To avoid running out of power, or activating different power saving measures, the charger was still connected
throughout the tests.

\paragraph{CPU power control}
\todo{Find a good source for explanation, possibly: https://www.quora.com/What-is-CPU-throttling, or talk with mariusz}
CPU frequency scaling is the situation where the frequency of the cpu is adjusted to fit the current computational needs,
in order to save power.
We want deterministic platforms when running our tests, hence we used the cpupower\footnote{\url{https://linux.die.net/man/1/cpupower}}
to better control the frequency of the CPU. To avoid issues with overheating, the frequency was set to \SI{2}{\giga\hertz},
which we considered a moderate amount.

\subsubsection{Emulating Behavior}
The different tests would need a way to mock behavior relevant to the test case.
These mocks were motivated by the lack of time to not create a proper realistic scenario.

\paragraph{Trivial Component}
\label{par:test_cases_trivial_component}
To mock processing within a component the trivial component was created.
The trivial component is a simple component structure which will do some work in the form of busy waiting,
and sending events to other trivial components.
The component will wait a certain time period to simulate computation, and potentially send an event
based on its waiting duration.
Other components will receive the event and print out a message describing the sender and receiver of the event.

\paragraph{Memory Usage}
While we needed to mock processing within the tests, we had enough components to make a realistic memory test.
Here we would use the NOX ECS implementation of the sprite component, which would be measured against
the NOX Actor implementation of the sprite component.

\subsection{Different Test Optimizations}
In addition to testing the system as a whole against the NOX Actor system, we wanted
to test some of the optimizations and choices in isolation.
This lead to the creation of some optimization switches.
These would be used to turn off or change implementation of certain behaviors within
the system.
When one of the optimizations were toggled the rest of the system was running with
default behavior, i.e. no two switches were enabled at the same time.
Running without any switches indicates default behavior.
\begin{itemize}
    \item NOX\_ECS\_COMPONENT\_VIRTUAL
    \item NOX\_ECS\_COMPONENT\_UNIQUE\_PTR\_VIRTUAL
    \item NOX\_USE\_STRING\_TYPE\_ID
    \item NOX\_ENTITYMANAGER\_USE\_LOCKED\_QUEUE
    \item NOX\_ENTITYMANAGER\_POOL\_USE\_LOCK\_FREE\_STACK
    \item NOX\_ECS\_LAYERED\_EXECUTION\_UPDATE
    \item NOX\_ECS\_LAYERED\_EXECUTION\_ENTITY\_EVENTS
    \item NOX\_EVENT\_USE\_LINEAR\_ALLOCATOR
    \item NOX\_EVENT\_USE\_HEAP\_ALLOCATOR
    \item NOX\_ATOMIC\_USE\_SEQ\_CST
    \item NOX\_LOCKFREESTACK\_USE\_LINEAR\_ALLOCATOR
\end{itemize}
The concepts of the switches are explained further in the following subsections.

\subsubsection{Atomic Sequential Consistency}
The NOX ECS uses relaxed atomics for its lock free implementations. We wanted to test
whether or not these were worth the implementation cost.
The tests were run on a strongly ordered architecture\cite{preshing_weak_vs_strong_memory_models}, where the
full potential of these relaxations are not shown. However, we still wanted to see
if we would gain any benefit using relaxed atomics versus sequentially consistent atomics.

\subsubsection{Numeric Type Identifiers}
The type identifier system offers two different ways to identify types, using
hashed strings, or using numeric values.
We wanted to inspect the performance difference between these to approaches.

\subsubsection{Allocators}
We have implemented 3 different allocators, the lock free allocator,
a locked linear allocator, and a heap allocator.
All three of these will be tested for the event arguments, while only the
linear allocator and lock free allocator will be tested with the lock free stack.
The reason for only using the two allocators on the stack is that it requires
a linear allocator to avoid the ABA problem\footnote{Explained in p.\pageref{subpar:detailed_lock_free_allocator_aba}}.

\subsubsection{Locked vs Lock Free}
We wanted to see the difference in performance regarding the lock free containers
and more primitive locked containers.
Therefore, a switch was added to use a locked container for lifecycle transitions and event processing.
We also added a test for using a lock free stack with the thread pool.

\subsubsection{Collection}
Alternative implementations of the component collection was tested,
mainly to give more information on memory optimizations and the virtual
replacements.

\paragraph{Virtual Usage}
In the first of the collection tests, the collection will function as normal, with one exception.
Rather than using the operation types within the meta information structure,
the collection will do virtual function calls on its components.

\paragraph{Unique Pointers}
The other test will be a simpler implementation of the collection.
In this case all the components will be allocated on the heap, and managed
using std::unique\_ptrs.
Virtual functions will also be used, rather than the meta information,
as we are not working with contiguous ranges.
No partitioning is done with the components.

\subsubsection{Execution Layers}
By default the ECS uses the execution layers on frequently called functions,
which include the update step and the distribution of events.
We wanted to see the difference in performance when toggling this,
and therefore ran tests for each function where the execution layers were disabled.

\subsection{Dropped Tests}
Certain planned types of tests will not be covered in the thesis, either because they were not executed, or not relevant.
This includes the multi-threading support and thread safe logging test.
The multi-threading test was rendered irrelevant because it is already covered by the numerous unique components test.
The thread safe logging test was removed because it did not contribute to the overall theme of the thesis.
\todo{Is this ok justification?}

\subsection{Compilation Test}
\subsubsection{Rationale}
Suttung requested that the compilation times should not significantly increase\reqref{req:templates},
we therefore added a test that would ensure that that this requirement was met.

\subsubsection{Flaws}
There is one major flaw with this test, and it is that it can't be performed without heavy restructuring of the code base.
The ECS is tightly coupled with the Nox Engine in the sense that a significant amount of the original code is still used in the ECS.
This means that if you compile the ECS, you also have to compile the Nox Engine.
However, it would be very difficult and time consuming to extract exactly the code needed for the ECS to avoid this problem.
A way around this could be to first compile the Nox Engine and the ECS.
Then measure and compare the compilation duration for the nox tests and ecs tests.
Unfortunately we were unable to find a way to partially compile the project in this manner.

\subsubsection{Method}
The test would be performed by first clearing all old build files by erasing the build directory.
Then the compilation is run in conjunction with the bash shell command time to measure the duration.

\subsection{Fast Spawning}
\label{subsec:test_cases_fast_spawning}
\subsubsection{Rationale}
The fast spawning test was supposed to check if the NOX ECS could improve performance in situations where a game would need to quickly
spawn a high amount of actors.

\subsubsection{Flaws}
The main flaw with this test was that the NOX ECS had not implemented any creation heavy weight components, like a physics component.
While the sprite component we used for the test has to set up connections to the graphics module, it does not require any more complex
computations. The physics module on the other hand, might need to set up some more complex structures.

\subsubsection{Method}
\label{subsubsec:test_cases_fast_spawning_method}
The fast spawning test case is performed by spawning a number of actors as fast as possible, before deleting them all.
This process is either executed one or ten times, depending on the setting of the test.
The number of actors spawning ranges from 256 to 32768, the number of actors is doubled between each increment.

\subsubsection{Changes}
\label{subsubsec:test_cases_fast_spawning_changes}
The test case was executed closely to the original plan, with the exception of not involving empty actors.
We did not see any reason to compare the two implementations for this case, as we don't see any reason for creating empty actors in a real life scenario.
Additionally this test might look skewed, as the NOX ECS does not allocate any memory for empty entities, but rather increments an atomic integer.

\subsection{Numerous Unique Components}
\subsubsection{Rationale}
The numerous unique components test was created to determine how NOX ECS would deal with a large number of different components,
and if the multi-threading could help in making processing of the components more efficient.
It puts pressure on how many component types the system can handle, and pushes the event system, as every component is sending a large amount of events.

\subsubsection{Flaws}
A major flaw with the numerous unique components test is its use of a trivial component that prints out results.
Having the test relying on printouts could affect the data in different ways.
This oversight was fixed for the test with the different optimization switches.
Here the print statement was traded with an empty volatile for loop, which the compiler could not optimize away.
Unfortunately we did not have time to rerun the tests comparing NOX ECS and the NOX Actor system.

\subsubsection{Method}
The numerous unique component test uses the trivial component\secref{par:test_cases_trivial_component} to simulate the use of a large amount of components.
The number of component types are either 2, 20 or 200, based on the test parameters. The components are variations of the trivial component.
Each component is labeled with a number, ranging from 0 to the max number of component types, indicated by the test.
This number is used to decide how long a component sleeps in order to simulate processing, and whether or not it should send events.
The components are given different data access values used for the execution layer algorithm.
The data access values are distributed evenly in the following order.
\begin{itemize}
    \item 25\%: Independent
    \item 25\%: Read only with one read connection
    \item 25\%: Read only with two read connections
    \item 25\%: Unknown
\end{itemize}

\subsubsection{Changes}
This test was implemented according to the original test case plan.

\subsection{Memory Usage}
\paragraph{Rationale}
The memory usage test were designed to measure the use of memory within the NOX ECS and compare it to the NOX Actor system.
It was motivated by the requirement from Suttung of not escalating the memory usage by a significant amount\reqref{req:memory_usage}.

\paragraph{Flaws}
The main flaw with the test is that it does not measure the memory usage in an entirely realistic case.

\paragraph{Method}
This test was performed by creating a set number of actors, the range of actors is the same as within the fast spawning test\secref{subsubsec:test_cases_fast_spawning_method}.
The actors will contain sprite and transform components.
The test will run for a minimum of one second, before quitting.

\paragraph{Changes}
The main difference in this test compared to the test case plan, was that we only used sprite and transform components,
rather than all of the variations mentioned in the test case plan\todo{Ref appendix test case}.
This change was mainly motivated by the lack of time, additionally the same points mentioned in the fast spawning test also applies here\secref{subsubsec:test_cases_fast_spawning_changes}.

% \subsection{Thread Safe Logging}
% \paragraph{Rationale}
% Needed a way to ensure the attempt at making the old logging system thread safe does what it is supposed to.

% \paragraph{Flaws}
% No apparent flaws in the test case.
% This can mostly be contributed to the fact that the problem at hand was not very difficult.
% The addition of a mutex in the logging system made it thread safe, and the way to ensure it was, also provides a strong case.

% \paragraph{Method}
% Each thread in the test will constantly print out the letter a through z for a given amount of time.
% If the logger was not thread safe, there would be a inconsistency where the letter a starts again too early, because a thread has been cut off in the middle of printing.
% After the program has run for a sufficient amount of time, the output is checked to see if any such previously mentioned errors has occurred.
% If there is no such errors, the logger can be considered thread safe.

% \paragraph{Changes}
% This test was done drastically different, due to it becoming apparent how much easier it can be performed.
% First off, there is no need for a ton of components doing this work of sending messages to put stress on the logging system, a couple of threads sending messages is fine.
% In addition, instead of using a regex check to see if every line is correct, the test was simply ran once with a single thread so no races could occur.
% After that the actual test was run, and the result of the single and multi threaded tests were checked to be equal, and if they were the test was a success.
