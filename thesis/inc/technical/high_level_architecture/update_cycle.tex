\subsection{Update Cycle}
\label{subsec:high_level_update_cycle}
In addition to the lifecycle the new ECS also has an update cycle with specific steps.
These steps are by default arranged in a specific sequence, but can to a certain degree
be executed in other sequences if so desired,
so long as the new sequence does not invalidate lifecycle requirements of a component.
The rationale for having these steps is that it allows for batched updates,
leading to more cache friendly data traversal and access patterns.
The following sections describes the different stages in the update cycle,
and are listed in their default order.

\todo{Fix this so it does not look so hilariously ugly}
\todo{This should be a figure}
\todo{perhaps make it clearer that "all requested components", are different subsets, not necessarily the same components.}
\begin{description}
    \item
    [Distribute Logic Events]
    Distributes all regular NOX Logic events to different components.
    Allowing nox to continue using their event system for non component communication.

    \item
    [Update]
    Goes through all components on a per type basis and calls their respective update function.
    The order of which types are called is unspecified.

    \item
    [Distribute Entity Events]
    Distributes all the different entity events to different components.

    \item
    [Deactivate Components]
    Deactivates all requested components.

    \item
    [Hibernate Components]
    Hibernates all requested components.

    \item
    [Remove Components]
    Removes (i.e. calls destructor on) all requested components.

    \item
    [Create Components]
    Creates (i.e. calls constructor on) all requested components.

    \item
    [Awake Components]
    Awakes all requested components.

    \item
    [Activate Components]
    Activates all requested components.
\end{description}
