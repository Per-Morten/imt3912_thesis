\subsection{Event System}
NOX Engine already allows for decoupled communication through their own event system.
The NOX event system is used both for sending events through different subsystems of the engine, and for sending messages between the various actors in their actor model.
We decided that we would introduce our own event system, for a number of reasons.

\subsubsection{Decoupled Event System}
As previously mentioned the NOX Actor event system is an extension of the generalized event system used within the NOX Engine. 
This system is also used to communicate with other engine subsystems, for example, the physics system.
We argue that the system for inter-entity communication has different requirements than communication with engine subsystems.
Creating a new event system that is decoupled from the old one, also gives us greater control.
Especially when introducing multiple threads into the picture, it is important to avoid the possibility of breaking the old event system.
Creating our own event system would ensure that there was no chance of breaking the old event system, while also allowing us to make it fit better within our proposed ECS.

\subsubsection{Avoiding Virtuals}
Another reason for the introduction of another event system was based on the requirement of avoiding virtual functions\reqref{subsec:requirements_performance_virtual_functions}.
The existing event system is based on a listener type pattern, requiring objects that wanted to receive events to implement IEventListener interface.

\subsubsection{ECS Events}
The proposed event system within the NOX ECS would not be particularity complex, and would not support as many features as the existing event system.
The "system" would be a simple event queue, allowing events to be queued for future
processing. Events can either be directed at specific entities or broadcast to all existing entities.
The processing of the entity events is done in the distribute entity events stage of the update cycle\secref{subsec:high_level_update_cycle}.
Events are more closely examined in section \ref{subsec:detailed_event}.
