\subsection{Event}
\label{subsec:detailed_event}
The events are our solution to the problem of allowing for decoupled communication
between components.
When a component wants to communicate a certain happening\todo{find better word}
to either other components, or other entities in general, the component will send an
event.

\todo{Discuss the Event "Queue" here, or in the Entity Manager?}

\subsubsection{Explanation}
The event class is based on Gregory\cite[p. 938]{game_engine_architecture} event arguments,
which is presented as key value pairs of variants. While we chose to work with key value pairs,
we choose a linked list approach rather than variants.

The event object is a simple singly linked list of event arguments, allocated from
the lock free allocator\secref{subsec:detailed_lock_free_allocator}.
In addition to these arguments, an event also knows its own type (i.e. name),
and its sender and receiver id. All of this information can be queried for.

\paragraph{Argument}
The event arguments non owning memory containers, containing its own identifier,
a pointer to the next argument, a pointer to a destructor and a pointer to its own payload.

\subparagraph{Destructor}
While the argument is not owning the memory of its payload, it is responsible for the object
in the payload getting destructed properly.
This is achieved in the same fashion as the operation types\secref{subsec:detailed_operation_types},
through a simple function pointer to a destructor.

\subparagraph{Get value}
A challenge when it comes to the argument is, once we throw out type information, how do we
get access to the value stored within an argument.
The only real way to do this is to give access to the payload of an argument and cast it
to the correct type. Currently this can be in two ways, by either taking a copy of the argument,
or by pointing directly into the payload.

\subsubsection{Motivation}
There were several factors that motivated the current implementation.

\paragraph{Decoupled Communication}
The main motivation behind the event system was to allow for efficient decoupled
communication between components.
It was desirable to avoid direct communication between components, as
that would inhibit concurrency and increase coupling within the system.
The rationale for creating a new event system, was also to remove the coupling
to other NOX Modules, as the old NOX Actor system was relying on the regular NOX Events.

\paragraph{Broadcasting Of Events}
Additionally there was a desire for being able for broadcast events,
sending them not only to a single receiving entity, but throughout the entire
system.

\paragraph{Fast Integration of New Events}
The last main motivation for this structure was a desire for being able to rapidly introduce new events.
Introducing events within the old NOX Actor event system required inheriting from a common base class,
meaning that one would have to create a new class for each type of event.
We wanted it to be easy to create new events within the new ECS, and without the requirement for creating
a new class. One should simply just be able to create an event with a certain type, add data to it,
and send it off.

\subsubsection{Alternatives to Event}
Several alternatives could have been chosen instead of this one.

\paragraph{Observer Pattern}
One alternative could be to use an observer pattern as described by Nystrom\cite[Observer]{game_programming_patterns},
which can be optimized to run quite fast.
One of the reasons for not going for this pattern was to avoid virtual functions,
and it would make the program harder to run concurrently.
It would also make it harder to run concurrently,
as this pattern quickly could introduce inter thread dependencies
that could be harder to detect.
Also it is synchronous in nature which further inhibits parallelism.

\paragraph{Use NOX Actor Events}
We could also possibly just reuse the regular NOX Actor events,
however this would lead to higher coupling internal within the NOX Engine.
It also would not fix a lot of the motivations behind the event system,
like the fast integration of events.

\paragraph{Direct Communication}
Another other alternative would be to drop the event system all together and
just allow for direct component collection. Obviously that would not be particularly
efficient, and would also make parallelism way harder than with the current solution.

\subsubsection{Pros of Event}
The main pros of the events is that it is quick to create new types of events.
\todo{Write more here, should be more pros, depends on whether or not we are writing about the queue her as well}.

\subsubsection{Cons of Event}
There are however several cons to the current event implementation.

\paragraph{Loss of Type Safety}
Since the events are essentially implemented as void* to avoid templates,
type safety is gone. When getting the values of object it is the programmers
responsibility to cast to the correct type.
However, this should not really be that big of an issue, because when a programmer
is getting the value from an argument, they do know the type of that argument,
and the type of the event. So the flexibility is worth the cost.

\paragraph{Maximum Sized Objects}
Unlike in the old NOX Event system which allocated everything from the heap.
The current event approach uses the lock-free allocator\secref{subsec:detailed_lock_free_allocator}
for allocating its arguments.
This means that there is a maximum size that is supported by the allocator.
However, this should not be an issue, as events are usually smaller primitive like types,
and the current lock-free allocator is templated on the size of its memory regions,
so if a larger size is needed one simply needs to adjust the template parameter.

\paragraph{Arguments Not Local}
Arguments within the event are not necessarily stored close to each other in memory,
which would be optimal for when they are processed. This is because of the linked list
implementation of the lock-free allocator.
However most arguments should be stored relatively close to each other, unless they are
placed within different memory blocks.
