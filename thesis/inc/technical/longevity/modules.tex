\subsection{Modules}
\label{subsec:longevity_modules}
As explained earlier in this thesis\secref{sec:requirements_templates}, C++ is known
for its long compile times, especially with heavy template usage.
As a result of this, as well as certain other factors,
there has been a desire to find more building mechanisms, like modules.
Among the goals of Modules is to improve C++ compile times.
Some of the ideas behind modules include the possibility to avoid parsing
template classes that are not instantiated, which in general would give better throughput\cite[49:53]{cppcon_modules_state_of_the_union}.
Modules are currently being experimented with by different compiler vendors
like Microsoft\cite{microsoft_modules_support} and Clang\cite{clang_modules_support},
and will not be adapted into the standard until earliest C++20, seeing as C++17 is feature complete\cite{isocpp_standardization_status}.
\todo{Find out if the microsoft clang thing should be sources, or just footnotes.}

\paragraph{Consequence}
If the modules functionality greatly decreases compile times in heavy templated code,
then there is no reason for the requirement of avoiding templates.
In this case the ECS should probably be moved over to preferring more template solutions,
which would allow for better optimizations based on type information, as well as supporting
proper typesafety.
Being allowed to use templates would change a lot of the implementation, examples being.

\subparagraph{Meta Information}
Meta information\secref{subsec:detailed_meta_information} and operation types\secref{subsec:detailed_operation_types}
would not be needed, as we could template most of our structures and use regular C++ syntax,
with the compiler instantiating all of our template classes at compile time.
Data dependencies would still be needed for the layered execution model.
However, modules would not fix all the problems with heavily templated code,
for example incomprehensible error messages.

\subparagraph{Component Collection}
Implementing the component collection with templates would be much more trivial.
The error prone byte pointer increment model would not be needed, as the compiler
and language would be able to help.
The collection could simply be implemented with std vectors, allowing for more
efficient implementations.
Potentially it could also be replaced with other third party solutions like
the plf colony\footnote{\url{http://plflib.org/colony.htm}}.

\subparagraph{Component}
In the current implementation of the ecs one of the reasons for having a component
base class is to give the containers an interface to work with, allowing for
generic code without properly knowing a components type.
With templates this need disappears, and the system could potentially move
over to a more duck typed library.
Where components are any object that implement a certain number of methods,
in our case this would be methods for the different lifecycle stages.
