\section{Sequentially Consistent Architectures}
\label{subsec:longevity_sequentially_consistent_architectures}
One of the reasons for creating the lock free containers, stack and allocator,
with relaxed memory models was to cater to architectures with a weak memory model,
such as the ARM architectures which are used a lot in mobile phones.
However, as noted by Sutter\cite[55:15]{herb_sutter_atomic_weapons},
software has converged on a sequentially consistent memory model,
and hardware is starting to follow suit, starting to implement sequentially
consistent memory models.

\subsection{Consequence}
With hardware moving over to sequentially consistent memory models,
or potentially stronger memory models, like x86\cite{preshing_weak_vs_strong_memory_models},
the rationale for continuing to use relaxed atomics disappear.
This would not have a large impact on the current lock free
structures, except that we can change all the memory orders
to being sequentially consistent, which has the benefit of being
easier to understand, and less error prone, as compiler and cpu reordering
essentially disappear.
