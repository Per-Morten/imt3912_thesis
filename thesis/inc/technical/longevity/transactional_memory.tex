\subsection{Transactional Memory}
\label{subsec:longevity_transactional_memory}
Currently the only two ways of synchronizing concurrent code in C++
is using locks, or atomics. Both of these alternatives have issues.
Locks for example open up the possibility of locking up the system,
while atomics are fairly challenging to program with.
Transactional memory is another way synchronize memory across different threads.
A transaction is a block of code which is executed in an atomic fashion.
In this case the atomic fashion means that within a transaction block,
a programmer sees memory as consistent within that block,
it also means that when a transaction is finished, all operations done within
the transaction are published to the rest of the program as a whole.
If two transactions happen at the same time on two different threads,
then one of the transactions might have to be retried, as the transaction
was based on an expected view of the memory that is no longer correct\cite{brett_hall_transactional_memory_in_practice}.
Currently transactional memory is not part of standard C++, but it is
in a technical specification phase.
Whether or not transactional memory will make some of our design choices
irrelevant mainly depends on the performance of the implementation of transactional
memory.

\paragraph{Consequence}
If transactional memory were to be implemented in an efficient manner, that would
be suitable to use within the requirements of games.
If that were the case then mainly the lock free data structures would be changed within
the ECS. Both the allocator and stack would be moved over to using transactional memory
rather than atomics, which would probably make the two containers easier to maintain.
