\chapter{Development Process}
\label{chap:process}

\subimport{process/}{plan.tex}
\subimport{process/}{roles.tex}
\subimport{process/}{tools.tex}
\subimport{process/}{scrum_process.tex}

\section{Criticism}
The process does have some problems that could have been executed in a better manner.

\subsection{Late Start}
We went through several different phases of deciding on what the thesis should actually be about,
this meant that we did not get around to starting working with the project as early as we originally
wanted.

\subsection{Division of Labor}
There was a clear division of labor within the project, with Per-Morten mainly developing
the new ECS while Trond were implementing the benchmark test cases.
This was probably not the best division of labor,
as we both ended up working quite a lot in isolation.
Having a more shared division of labor would probably open us both more to critique,
leading to an overall better product.
It would probably also have allowed us to find the issues within the test cases at an earlier stage,
and made them more representable.
The clear division of labor was mainly done to make better use of the time we had available.

\subsection{Poor Execution of Plans}
We did spend some time planning the development before we started, however we were unable
to execute the plans in a desirable fashion. Additionally we often realized this to late,
and were unable to respond quickly enough to the new changes while simultaneously updating the plans.

\subsection{Optimization Without Profiling}
Due to the lack of time, we were unable to profile our code while optimizing it.
This is not particularly good development practice, as we don't have data to base
our decisions on.
The fact that we were not able to profile our code properly while optimizing it
is one of the reasons for including the internal tests, as these give us some
indications on which optimization that was worth it and not.
