\chapter{Future Work}
\label{chap:future_work}
Unfortunately the development of the NOX ECS did not get as far as we originally hoped.
The following sections details areas that we would like to improve upon.

\section{Code Improvements}
\subsection{Remaining Features}
\subsubsection{Lifecycle Queries}
Currently there are still some features missing from the NOX ECS, like being able to query
components for their lifecycle stage, which was possible in the original Actor model.
Implementing this functionality should be trivial.

\subsubsection{JSON Integration}
The factory functionality for loading and creating entities through JSON objects
is not equivalent to the functionality found in the factory of the NOX Actor system.
This needs to be added before the NOX ECS can be properly used by Suttung.
This functionality also include proper serialization with JSON objects, which
is currently not supported in NOX ECS.

\subsubsection{Alternative Implementations}
A lot of the implementations covered in the detailed architecture section mentioned alternative
implementations. Some of these alternative solutions should be pursued in the search for
better performance and cleaner implementations. For example, the external references functionality
mentioned in the component collection.

\subsection{Safety}
A requirement from Suttung was a focus on safety, and encouraging correct use of API's.
This requirement was not considered as much as it should have been, and is something that
should be improved upon in the future.
For example, the type identifier is not particularly user friendly, and mechanisms should
be put in place to ensure correct usage of this class.

\section{Benchmarking}
\subsection{Mobile Tests}
Mobile benchmarking tests were not performed with the NOX ECS,
the benchmarks should be run on a mobile platform to give information about performance.

\subsection{Multiple Compiler Tests}
Currently the benchmark tests were only run on GCC. However, the run test scripts are also
set up for allowing tests to run on Clang, they were just not run because of our lack of time.
The benchmark test should also be run on Clang, to check if we get the same results.
Additionally, the benchmark tests should also be run on other compilers for different platforms,
like Microsoft Windows.

\subsection{Template Tests}
While a lot of the individual optimizations have been tested, like using virtual functions rather than
operation types. However, we did not have time to create a template based entity component system.
Having such an example would allow us to see what gains we could get out of using templates.

