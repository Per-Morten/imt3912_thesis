\chapter{Future Work}
\label{chap:future_work}
Unfortunately the development of the NOX ECS did not get as far as we originally hoped.
The following sections details areas that we would like to improve upon.

\section{Code Improvements}
\subsection{Remaining Features}
\subsubsection{Lifecycle Queries}
Currently there are still some features missing from the NOX ECS, like being able to query
components for their lifecycle stage, which was possible in the original Actor model.
Implementing this functionality should be trivial.

\subsubsection{JSON Integration}
The factory functionality for loading and creating entities through JSON objects
is not equivalent to the functionality found in the factory of the NOX Actor system.
This needs to be added before the NOX ECS can be properly used by Suttung.
This functionality also include proper serialization with JSON objects, which
is currently not supported in NOX ECS.

\subsubsection{Alternative Implementations}
A lot of the implementations covered in the detailed architecture section mentioned alternative
implementations. Some of these alternative solutions should be pursued in the search for
better performance and cleaner implementations. For example, the external references functionality
mentioned in the component collection\footnote{See p.\pageref{subpar:detailed_component_collection_external_references}}.

\subsection{Safety}
A requirement from Suttung was a focus on safety, and encouraging correct use of API's.
This requirement was not considered as much as it should have been, and is something that
should be improved upon in the future.
For example, the type identifier is not particularly user friendly, and mechanisms should
be put in place to ensure correct usage of this class.
A solution could be to create a script that analyzed the different usages of the type identifiers,
for example detecting if the reserved identifiers were overridden by user provided ones.
Additionally some extra assertions should be added to defend again user errors disturbing internal
invariants within the ECS.

\subsection{Data Access Detection Script}
Writing the data access inn by hand to make proper use of the execution layers are relatively time consuming
and error prone.
It would be a better solution if we could write a script that would analyze the different
components and find their dependencies\footnote{See p.\pageref{par:detailed_execution_layers_code_analysis}}.
A script could also allow for pre-computation of the execution layers, which could be stored on file,
rather than a computation at runtime.
Such a script could potentially also inform users in the cases where there would be uncertainties related
to the benefits of using the execution layers.
Like if a lot of components would need to be run single threaded.

\subsection{Standard Components}
The NOX Actor system does offer a number of standard components, for example lighting components,
controller components etc.
Additionally NOX ECS needs to offer some extra components to make up for some lacking functionality,
for example a component that contains more type information, like what sort of json object the entity was created form.
A physics component should also be added, however this will require a larger development effort, as the physics logic
is quite tightly coupled to the NOX Actor system.

\subsection{Various Improvements}
There are also a various of improvements that should be done to the code base which does not warrant
their own sections. Examples include proper utilization of the main thread within the thread pool,
or better utilization of type information, like removing unnecessary destructor calls on trivial types.

\subsection{Proper NOX Integration}
The NOX ECS still have some way to go before it can be properly tied into the NOX Engine.
For example the entity manager does not have any mechanism for listening to regular NOX events.

\section{Benchmarking}
\subsection{Fixing Criticism}
\todo{Find better title}
There were several points that can be made to critique our findings\secref{sec:discussion_criticism}.
In the future these shortcomings should be examined, and ideally be eliminated.
This means that initially test should:
\begin{itemize}
    \item
    Run on different compilers.
    \item
    Run on different PC platforms.
    \item
    Be repeated to verify our findings.
\end{itemize}
Fixing these issues should not be a large problem, for example our testing scripts already include
running tests on Clang. This functionality was turned off because of the time constraints.
Additionally there are several benchmarking issues that should be fixed.

\subsection{Mobile Testing}
Testing on mobile was originally a planned, but this was never accomplished due to lack of time.
This would have been a good way to check if the lock free structures with relaxed atomics actually
gave an increased boost in performance.

\subsection{Template Tests}
A lot of the individual optimizations have been tested, like using virtual functions rather than
operation types. However, we did not have time to create a template based entity component system.
Having such an example would allow us to see what gains we could get out of using templates,
and the optimizations that the compiler could do based on type information.

\section{User Tests}
We had originally planned to have user tests together with Suttung.
The idea was to arrange a game jam with Suttung over a weekend,
and do tests with them to gather feedback on the new system.
Unfortunately neither we nor Suttung were able to dedicate the time to this when the system
was in a testable state.
This is something that should be done in the future, as it would allow us to gather
data on how the new NOX ECS is to work with, and if there is a large initial loss of productivity.
