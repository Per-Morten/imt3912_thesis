\section{Game Industry}
As developers of quality interactive experiences, modern video game developers face a fairly unique
combination of challenges.

\subsection{Flexibility}
A valued concept within the video game industry is fast iteration cycles.
Game developers often want to be able to test out and iterate on their ideas quickly.
Fast iteration cycles require a great deal of flexibility.
It is therefore desirable to keep software modules decoupled and generic,
allowing developers to quickly combine different modules into new combinations creating new functionality,
and content.

\subsection{Reusability}
Similar are the desire for reusability.
Like many other branches of software, the game industry are interested in reusable software modules, to save both money and development time.

\subsection{Performance}
Jason Gregory\cite[p. 9]{game_engine_architecture} describes video games as "soft real-time interactive agent-based computer simulations".
As with all real-time systems certain deadlines need to be met, the most obvious example being frame rate requirements.
Video games are also expected to push limits, potentially for multiple platforms with different strengths and weaknesses,
constantly improving areas like graphics, physics simulations, AI, etc.

These desires are often at odds with each other, and finding a balance is a difficult task.
Additional factors also affect the balancing of these desires, like the type of game that is being developed,
or what platforms the game is being developed for.

\subsection{Game Engines}
Modern video games are often created using game engines. A game engine can be seen as a foundation or framework
containing a set of core modules needed for a game. Examples of core modules are rendering or physics modules.
Different game engines have different strengths and weaknesses, depending on different factors.
For example what sort of game genre the engine was designed for.
The use of game engines can allow developers to create games at a faster pace and lower costs.
\todo{Do we need sources for this?}
