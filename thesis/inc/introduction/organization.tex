\section{Organization}
The document is split several chapters, with appendices at the end.
\begin{enumerate}
    \item
    \hyperref[chap:introduction]{Introduction}: Introduction discusses the purpose of the thesis, and introduces the problem domain.
    \item
    \hyperref[chap:requirements]{Requirements}: Requirements describes the requirements agreed upon by Suttung and the authors,
    with some motivation to why the requirements are in place.
    \item
    \hyperref[chap:technical_high_level_architecture]{Technical - High Level Architecture}: High Level Architecture gives a brief overview of the different parts of the proposed NOX ECS.
    \item
    \hyperref[chap:technical_detailed_architecture]{Technical - Detailed Architecture}: Detailed Architecture describes the modules of NOX ECS in detail, including implementation where applicable.
    \item
    \hyperref[chap:technical_longevity]{Technical - Longevity}: Longevity discusses possible future expansion to the C++ language, rendering some of the implementation choices possibly irrelevant.
    \item
    \hyperref[chap:measurements]{Measurements}: Measurements describes how we will measure performance through benchmark tests.
    \item
    \hyperref[chap:benchmarking]{Benchmarking}: Benchmarking describes the findings of running the performance tests, and gives some speculation on the causes of the results.
    \item
    \hyperref[chap:discussion]{Discussion}: Discussion generalizes the results found in the benchmarking chapter, and discusses them on a broader scope.
    \item
    \hyperref[chap:future_work]{Future Work}: Future work identifies areas of improvement for both the NOX ECS and the tests that were conducted.
    \item
    \hyperref[chap:process]{Process}: Process discusses the development of the NOX ECS.
    \item
    \hyperref[chap:conclusion]{Conclusion}: Concludes the thesis, and evaluates the project.
\end{enumerate}
