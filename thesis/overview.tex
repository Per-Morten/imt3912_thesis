High level overview of the document.
Everything written here will obviously be explained further in the proper thesis, this is just to try and give a general skeleton.
Section and chapter names might not be exactly the same either,
however the flow will be like this outline.

We have chosen to split the document a bit more to have a larger difference on what
that is related to the benchmark results etc, and how we worked.
Up until the process chapter, we are not really talking anything about how we worked and planned,
but rather what we are researching and the results of our implementation.
The process chapter is where we start talking about how we were doing planning,
scrum models, changes during development etc etc.

%%%%%%%%%%%%%%%%%%%%%%%%%%%%%%%%%%%%%%%%%%%%%%%%%%%%%%%%%%%
%%% Introduction
%%%%%%%%%%%%%%%%%%%%%%%%%%%%%%%%%%%%%%%%%%%%%%%%%%%%%%%%%%%
\chapter{Introduction}
\label{chap:introduction}

\section{Game Industry}
Short high level description of the issues faced within the games industry: fast iteration cycles, performance.

\subsection{Fast Iteration Cycles}
Why does the game industry need fast iteration cycles?
Flexible design, need to test ideas quick etc.

\subsection{Performance}
Why does the game industry value performance?
Always pushing the envelope for graphics etc.

\subsection{Flexibility}
Why does the game industry value flexible systems?
Reuse existing stuff.

\section{Entity Component Systems}
High level overview over what entity component systems are, and why they are used in the games industry.

\section{Introduction To Suttung}
Introduction to Suttung, who is our stakeholder.

\section{Introduction To NOX Engine}
Short introduction to the NOX engine, which is what we will interface against.

\section{High-Level Objective}
Create an Entity Component System within Suttungs engine,
based on the requirements laid down by Suttung.

\section{Organization}
How is this document organized.
How we organize ourselves, process etc will be put later in the document.

\section{Terminology}
A small section explaining what the different terms used within the report means.

%%%%%%%%%%%%%%%%%%%%%%%%%%%%%%%%%%%%%%%%%%%%%%%%%%%%%%%%%%%
%%% Requirements
%%%%%%%%%%%%%%%%%%%%%%%%%%%%%%%%%%%%%%%%%%%%%%%%%%%%%%%%%%%
\chapter{Requirements}
\label{chap:requirements}

\section{Templates}
Explain why we don't want heavy template usage within our system.
To avoid heavy compile times within our ecs.

\section{Performance}
Discuss quickly why performance is important for NOX.

\subsection{Virtual Functions}
Discuss how virtual often are implemented, and why that layer of indirection is not necessarily good for modern platforms.

\subsection{Contiguous Memory}
Discuss why contiguous memory is important on modern platforms.

\subsection{Multi-threading}
Discuss why nox would like multi-threading within the new ecs.

\section{Memory Usage}
Discuss how we wanted to keep the memory usage about the same.

\section{Mobile Platform}
Discuss how mobile platforms often aren't x86 architectures, and what that affects.
Mention both memory models, as well as the use of memory.

\section{Familiarity}
Discuss the requirement from NOX of familiarity, that the new system should not in a great deal different
than the one they already have.

\section{Focus On Safety}
Discuss how nox would prefer us to do stuff safely but performance pessimistic by default.

\section{Final Problem Statement And Objective}
Here we detail the actual problem statement and objective based on the requirements listed above.

