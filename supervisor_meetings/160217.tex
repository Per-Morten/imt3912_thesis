\documentclass[hidelinks]{article}

\usepackage{hyperref}

\begin{document}

\begin{center}
\subsection*{Supervisor meeting 16.02.2017}
\subsubsection*{With: Mariusz Nowostawski}
\end{center}

\bigskip


\subsubsection*{Intention of the meeting}
Update, and discussion on time.


\subsubsection*{Questions to the supervisor}
\begin{itemize}
    \item 
    Is it ok to change to only having one iteration, as the scope of the project is getting larger.

    \item
    Quick feedback on design document draft.

    \item
    Are we supposed to write status reports?\\ 
    Other than our shortly written regular meetings.\\
    Example: \href{https://fronter.com/hig/links/files.phtml/1602890066$1063371470$/FagstoffRessurser/02+Viktige+dokumenter+du+trenger/Retningslinjer+bacheloroppgave+statusrapport.pdf}{Link}
\end{itemize}


\subsubsection*{Answers to our questions}
\begin{itemize}
    \item
    Write amendments to the original plan, however argument for why they are happening.
    Re-scope as early as possible.

    \item
    Feedback on design document and thoughts on proposed design. 
    \begin{itemize}
        \item
        Clarify more with Suttung focus on maintainability vs performance.
    
        \item
        Think of creating smart pointer like functionality, that deallocate components at end of scope.
    
        \item
        Think of how we can add extra security features. 
    
        \item
        Go for this system, but perhaps with some of Suttung's revisions. 
    
        \item
        The discussion on balance of flexibility vs performance vs maintainability is important for the thesis.
    
        \item
        We can discuss how we think that the balance is achieved through our opt-out system.
    \end{itemize}

    \item
    Write this sort of document after sprinting or when re-scoping.
    \begin{itemize}
        \item
        Be more stringent on actually having an iterative development cycle.
    
        \item
        Continue with 1 week sprints, revise plan each 2 sprint.
    
        \item
        Odd sprints carry over, while even sprints are forced closed.
    
        \item
        This also applies to point 1 about re-scoping.
    
        \item
        Outside of the sprint reviews and during re-scoping the current format of daily scrums works fine.
    \end{itemize}
\end{itemize}


\subsubsection*{New tasks from our supervisor}
Re-scope the project. Artificially stop the current "sprint" and re-plan.
Create a proper iterative development cycle.
Notify Simon and Mariusz when the design document is done.



\end{document}
