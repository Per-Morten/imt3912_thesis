\documentclass[hidelinks]{article}
\setlength\parindent{0pt}

\usepackage{tabularx}



\begin{document}

\begin{titlepage}
    \centering
    {\scshape\LARGE NOX ECS 2: Scrum Plan 3 \par}
    \vfill
    {\large \today\par}
\end{titlepage}

\tableofcontents
\pagebreak

\section{Rationale}
As a result of sickness, and miscalculation of time, we have again decided to re-scope and re-plan the different sprints.

\section{Goal}
The new goal remains much the same as the previous one, with the exception of removing state caching to deal with inter-thread dependencies.
This approach is not possible, as components are move only types. 
The Direct Acyclic Graph dependency solution has been renamed to Layered Execution, as we are not really doing a DAG structure.

\section{Sprints}
The following sections contains the new sprints. The names of the sprints are kept the same, so the Gant chart in the original project plan can still be used.
After each sprint we will have a review of the sprint that has just passed, and after each second sprint we do a re-evaluation of our forward plan.

Writing the thesis has not been done in a continuous manner, as we have been to busy implementing the system. 
Writing the thesis will begin either after all dev sprints are finished, or start getting ahead of our goals.

\subsection{Sprint 7}
Did not happen as a result of sickness.

\subsection{Sprint 8}
\textbf{Time: 20.03-24.03}\\
\begin{tabularx}{\textwidth}{l >{\centering\arraybackslash}X}
\hline\noalign{\smallskip}
  Task                              & Testing Requirements   \\
\hline\noalign{\smallskip}
  \textbf{Memory Efficiency}:       &                        \\
  Event Allocator                   & High                   \\
  Event Argument Allocator          & High                   \\
  
  \textbf{Dependency Problem}:      &                        \\ 
  Layered Execution                 & High                   \\

  \textbf{Write Test Cases: NOX}:   & N/A                    \\
  Compilation Time                                           \\

  \textbf{Run Test Cases: NOX}:     &                        \\
  Compilation Time                  & N/A                    \\
  Fast Spawning                     & N/A                    \\
  Numerous Unique Components        & N/A                    \\
  Memory Usage                      & N/A                    \\
  Multi-threading Support           & N/A                    \\

  \textbf{ECS: Standard Components} &                        \\
  Transform                         & Med                    \\
  Graphics                          & Med                    \\
  Physics                           & Med                    \\

\hline\noalign{\smallskip}
\end{tabularx}

\subsection{Sprint 9}
\textbf{Time: 27.03-31.03}\\
\begin{tabularx}{\textwidth}{l >{\centering\arraybackslash}X}
\hline\noalign{\smallskip}
  Task                                  & Testing Requirements   \\
\hline\noalign{\smallskip}
  \textbf{ECS: Standard Components}     &                        \\
  Json Type Information                 & Med                    \\

  \textbf{Write Test Cases: ECS}:       &                        \\
  Compilation Time                      & N/A                    \\
  Fast Spawning                         & N/A                    \\
  Numerous Unique Components            & N/A                    \\
  Memory Usage                          & N/A                    \\
  Multi-threading Support               & N/A                    \\

  \textbf{Run Test Cases: ECS}:         &                        \\
  Compilation Time                      & N/A                    \\
  Fast Spawning                         & N/A                    \\
  Numerous Unique Components            & N/A                    \\
  Memory Usage                          & N/A                    \\
  Multi-threading Support               & N/A                    \\

  \textbf{Lock Free Multi-threading}:   &                        \\
  Block Allocator                       & Very High              \\
  Queue                                 & Very High              \\
\hline\noalign{\smallskip}
\end{tabularx}

\subsection{Sprint 10}
\textbf{Time: 03.04-07.04}\\
Used as a buffer to finish up, or starting early on thesis.

\end{document}
