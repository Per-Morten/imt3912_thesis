\subsection{Arguments for development model}
For the development of our bachelor project, we decided to go for  scrum, together with certain practices borrowed from XP and RUP that we feel are appropriate.


We want our development model to produce a sufficient amount of artifacts and documentation for writing the thesis after the development process is finished.
In addition, our previous knowledge of scrum allows us to start utilizing the model much faster, and everyone is up to speed on what to do from the get go.
Theres also the fact that we have access to JIRA, which provides us with all the scrum utility tools in one place, making the whole process much easier.


\subsubsection*{Why not eXtreme programming?}

The first major issue we had considering XP was the lack of artifacts and documentation produced by the process.
We could have made it work by changing the model around to include more planning and documenting, but that would move the model away from what it is supposed to achieve, which is fluid development throughout the process with new requirements added, changed and removed the entire time.
In addition we need a plan to show Suttung that they can approve, in order for us to know if we are on the right track.


\subsubsection*{Why not RUP?}

Even though the whole inception, elaboration, construction and transition phases fits quite well with our project, we feel that this is on the opposite side of the spectrum compared to XP when it comes to amount of documentation and artifacts. 
The amount of overhead caused by RUP would be too much for a development team with only two members.
Use-cases and business models are important in the RUP model, but both are not worth investing our time into.
Use-cases is a bit over the top for us, and creating business model is not useful.
If we wanted to make RUP fit our requirements, we would remove a lot of its features, making it into something resembling scrum, hence why not just choose scrum.


\subsubsection*{Why not Waterfall model?}

A bit on the fence about going agile or not, since our project does require a substantial amount of research and planing before we can start the development, and there is a hard deadline for the thesis making a rigid plan plausible to utilize. 
Uncertainties with the specific requirements both from not knowing what kind of entity component system we are gonna make, and how it should be structured to fit into the NOX-Engine means that we cannot use the waterfall model.
It would've required way too much time in the planning phase to completely understand everything needed to implement our system.