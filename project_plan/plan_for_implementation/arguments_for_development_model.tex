\subsection{Arguments for development model}
For the development of our bachelor project, we decided to go for a lite version of scum, together with certain practices borrowed from XP and RUP that we feel are appropriate.


We want our development model to produce a sufficient amount of artifacts and documentation for writing the thesis after the development process is finished.
In addition, our previous knowledge of scrum allows us to start utilizing the model much faster, and everyone is up to speed on what to do from the get go.
Theres also the fact that we have access to JIRA, which provides us with all the scrum utility tools in one place, making the whole process much easier.


We also use a number of practices from XP and RUP in our development model. 
\begin{itemize}

    \item Coding standard


    The coding standard is there to give a consistency throughout our code, making it easier for the team to read others code, and also come to an agreement on what is right and wrong ways to write certain bits of code.


    \item Collective code ownership


    Everyone should have as much insight in what happens in the entire project at all times.
    This is going to be solved with weekly code reviews, explaining newly implemented code to the rest of the team.


    \item Simple design


    In every aspect of the project, we will have the mindset of trying to create a new functionality as simple as possible. 


    \item 

\end{itemize}

\subsubsection*{Why not eXtreme programming?}
Too few artifacts
Pair programming not that useful for a two man team
Disregards planning, which we need


\subsubsection*{Why not RUP?}
Too much overhead
Use-cases are not as valuable for us, not really any use-cases for us
Business models are useless for us, same with user manuals
If we go for this model, we feel that we are gonna remove too many features for it to work for us.
Might as well go with scrum, as we would in practice transform RUP into something closely related to scrum anyhow.


\subsubsection*{Why not Waterfall model?}

A bit on the fence about going agile or not, since our project does require a substantial amount of research and planing before we can start the development, and there is a hard deadline for the thesis making a rigid plan plausible to utilize. However, because of uncertainties regarding how much work is required in the planning phase, and what

\subsubsection*{}
Using "lite" version of scrum with certain practices borrowed from XP and RUP
Allows us to produce the artifacts and documentation we desire
Previous knowledge of scrum allows us to more easily utilize the model
XP's process is a bit lackluster when it comes to producing artifacts, and also wants to continuously develop your requirements along the way, something we are not going to do. It really is just too agile, their extreme is too extreme. Our plans and requirements are rigid enough for it to be unnecessary.
Uncertainties with the specific requirements both from not knowing what kind of entity component system we are gonna make, and how it should be structured to fit into the NOX-Engine means that we cannot use the waterfall model, which would've required way too much time in the planning phase to completely understand everything needed to implement our system.
Don't want too much overhead either, since we are only two people, something that RUP would cause.
Already have a nice system (jira) to do a lot of the scrum stuff for us.
