%Todo put these into their own file, or put then in the project plan.
\newcommand{\ptparagraph}[1]{\paragraph{#1}\mbox{}\\}
\newcommand{\ptsubparagraph}[1]{\mbox{}\\\textbf{#1}\mbox{}\\}

\section*{Identified Risks}
Bellow follows the identified risks related to this project, 
as well as the consequences related to the individual risks.

\subsection*{Poor performance}
Performance is a pretty hard topic, especially when combined with
systems as complicated as ECS's.\\
\textbf{Probability}: Medium\\
\textbf{Impact}: Medium

\ptparagraph{Consequence}
A poor performing system would mean that we would have to spend more time
optimizing the ECS, which could get in the way of the other tasks.
A suboptimal solution would probably not be something Suttung would be interested
in using either.

\subsection*{Incompatible Implementation}
The main goal of this project is to allow Suttung to easily integrate the new ECS
into their system. 
However it might be that our system ends up fundamentally different than theirs,
and that the integration is not an easy process.
\textbf{Probability}: Medium\\
\textbf{Impact}: High

\ptparagraph{Consequence} 
An incompatible implementation would mean that the Suttung group would not be able to
easily integrate the ECS into their systems without major changes. 
In this case Suttung might decide that the new ECS is not worth it,
and stay with their current solution.

\subsection*{Multi-threaded Related Bugs}
Writing multi-threaded code is a difficult challenge, and finding bugs in multi-threaded
code is even more difficult and time consuming.
\textbf{Probability}: High\\
\textbf{Impact}: Depends

\ptparagraph{Consequence}
The consequence of a multi-threaded related bug depends on what sort of bug, 
and how fatal it is. Non fatal bugs with low recurrence is obviously not wanted,
but it might be acceptable. This cannot be said for fatal program terminating bugs
that recur with a high frequency.

\subsection*{Unrepresentative Test Case}
The plan is to write a representative test case with the NOX Engine that we can use
to see the flexibility and usage of the engine. However Suttung might decide that this
case is not representative enough for their normal usage patterns.
\textbf{Probability}: Low\\
\textbf{Impact}: Medium

\ptparagraph{Consequence}
In the case where the test case is not representative enough, we would probably have to
rewrite the test case, which will then take up time that we don't really have available.

\subsection*{Lost Work}
Lost work could take many different forms, it could be loss of source code or documents.
\textbf{Probability}: Med\\
\textbf{Impact}: Depends

\ptparagraph{Consequences}
The consequence of loosing work would be to either redo it, or drop the features if possible.
How bad this is for the project depends on how much work that needs to be redone, and
if we have enough time to redo the work.