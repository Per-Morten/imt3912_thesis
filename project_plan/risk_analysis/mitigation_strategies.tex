\subsection{Mitigation Strategies}
Bellow follows our strategies for mitigating the chances of the identified risks happening.
And how we might deal with them if they would occur.

\ptparagraph{Poor Performance}
Performance problems will be mitigated through recurring profiling.
Benchmarks should ideally be run quite often,
and these benchmarks should guide further optimizations if needed.

\ptparagraph{Incompatible Implementation}
The implementation will first go through a design phase,
before it is presented to Suttung. 
This proposed design will be built on the research
that have been conducted into the Suttung engine,
as well as a relevant test case.
Hopefully this will be enough to mitigate the changes
of our implementation being incompatible with the NOX Engine.

\ptparagraph{Hard to find Bugs}
Reducing the risk of bugs is a difficult task, 
however some general guidelines helps when developing code.
For example,
we will have a focus on clear access patterns,
and local scope when writing multi-threaded code.
There might also be some libraries out there that
allows us to build tests specifically aimed at
multi-threaded code.
Other mitigation strategies includes writing more tests for the different code sections.

\ptparagraph{Unrepresentative Test Case}
The test case will be developed based on usage patterns
observed in real life use of the NOX Engine.
The plan is also to verify the case with Suttung.

\ptparagraph{Lost Work}
All of the source code will be uploaded to on-line repositories, 
this will also apply to the thesis itself, as well as
all other documents related to the project.
The group members are encouraged to commit work to these repositories often,
preferably after each discrete "unit of work".

\ptparagraph{Internal Conflict}
Both members of the group have worked together previously on project,
which has gone quite well. 
Because of this we have not added extra mitigation factors, 
outside of the group rules.

\ptparagraph{Problem Understanding the NOX Engine}
The best way to mitigate this risk is to use more time with the engine.
We will therefore write the test cases that we are
going to use for comparisons in the NOX Engine first.
Suttung will also give us access to one of their games which is built with the NOX Engine.
This will give us some clearer examples to look at, and hopefully reduce the time 
used to understand the engine.

\ptparagraph{Missing Group Members}
Missing group members is hard to mitigate, however 
we will through daily meetings keep the other member
up to date on what we are working on. 
Hopefully this will lessen the impact if a group member
does not show up one day.
In addition group members are encouraged to work remotely when possible,
if they for some reason cannot show up for work.

\ptparagraph{Lack of Proper Measurement Tools}
There are not that many performance measurement tools
directed at C++ out there. 
We have already found a few profilers with a high precision level. 
However if we are not able to get the data we need out of these profilers,
we might have to go to less precise measurement methods,
like measuring frame rate.

\ptparagraph{Wrongly Estimated Time}
We have already mitigated the consequences of this risk a bit, by including two weeks of buffer time.
Planned regular meetings will also help reducing this risk, as it allows us to keep 
track of how we use our time, and potentially scale back some features or tasks.