\section*{Background}
The video game industry is one of the largest entertainment industries in the market today. 
With a focus on creating unique experiences as well as pushing the performance and graphics fidelity
forward the video game industry have encountered a plethora of technological challenges.
The technology powering these games must be flexible enough to allow for creative gameplay situations, 
while still able to meet the stringent performance requirements of modern games.
This balance issue of flexibility vs performance have lead to some interesting patterns within the industry.

\subsection*{Entity Component Systems}
Entity Component Systems originally appeared towards the end of the 1990's, 
however the popularity of this approach has only grown. 
\cite{wikipedia_ecs_history}
Entity Component Systems exists in various forms with different flavors,
however they all seem to share the same backbone, the focus on domain specific, 
decoupled and isolated behavior building blocks.
These building blocks can then be combined in almost endless fashions to create complex behavior. 
The goal of an entity component system is to reduce complexity and decoupling within games. 
This in turn leads to greater flexibility, higher code reuse, and shorter iteration cycles. 
An entity is often defined as something that exists in the game world. 
How the entity functions is described through the component it contains, 
i.e. there is a larger focus on aggregation rather than inheritance. 
\cite[components]{game_programming_patterns}
As mentioned earlier there are several forms of entity component systems. 
Some implementations keep a more traditional OOP implementation, 
\cite[components]{game_programming_patterns}
while others are more akin to relational databases, 
where entities are simply identified through a unique ID. 
\cite{t_machine_ecs_are_the_future_p2} 

\subsection*{Data Oriented Design}
Data Oriented Design(DOD) represents a shift in perspective in relation to software and modeling. 
Traditional Object Oriented Design favors layers and layers of abstraction and focus on single elements, 
DOD takes a completely different route. 
DOD focuses on the data that will flow through the program, what sort of data is it, 
its frequency, how its transformed, what patterns does it follow, etc. 
The architectural decisions when designing the program are made to ensure that the flow of data
 and transformations happens as efficiently as possible.
\cite{noel_dod_shoot_in_foot, dod_com} 
The DOD focus on data is also influenced by the growing gap between memory and CPU speed. 
The fact that memory access is a huge bottleneck within games, lead DOD down the road of optimizing for memory, 
with a focus on homogeneous transformations on large sets of data at the time.
\cite{pitfalls_of_oop}

\subsection*{NOX Engine}
The NOX Engine is an open source game engine originally developed by former students at Gj{\o}vik University College.
The engine is still in use today by the company Suttung Digital. 
After several discussions with Suttung Digital regarding the aim of this project, 
it was decided that we would develop a new entity component system for their engine. 
This entity component system would have a focus on performance, 
support multi-threading and have a data-oriented mindset.