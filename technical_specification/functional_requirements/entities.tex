\subsection{Actors/Entities}
An actor, henceforth called an entity describes an object within the game world. 
Entities can be seen as containers of components, 
and their functionality is defined through the components they contain.
Actors can only contain one of each type of component, to contain "more" than one of each component, a child must be added. 
A child is just another entity, that is "linked" with the parent entity.

\ptparagraph{Initialization}
Entities can be initialized in one of two ways.
\begin{itemize}
    \item
    Through a json object.

    \item
    Directly in C++ code.
\end{itemize}

\ptparagraph{Queries}
Entities can be queried for the following information.
\begin{itemize}
    \item
    Information related to the type of the object.

    \item
    The entity's unique id.

    \item
    The entity's components.

    \item
    Children of the entity.

    \item
    An entity's parent.

    \item
    An entity's stage in the lifecycle.
\end{itemize}

\ptparagraph{Manipulation}
The following manipulations can be applied to an entity.
\begin{itemize}
    \item 
    Attach/Detaching children.

    \item
    Attach/Detaching components.
\end{itemize}

\ptparagraph{Lifecycle}
Actors will be made aware of and can react to the following events in their lifecycle.
\begin{itemize}
    \item
    An entity's own
    \begin{itemize}
        \item creation.
        \item activation.
        \item deactivation.
        \item destruction.
    \end{itemize}

    \item
    Per frame updates.
\end{itemize}
