\section{Measuring Quality Requirements}
To ensure that the quality requirements are fulfilled several test cases will be built. 
These test cases will be used to measure whether or not we have fulfilled the quality requirements.
This section outlines how the different requirements will be tested, and what metrics that will be used.
All tests will be written with both the current NOX Actor system, and the proposed ECS, and evaluated in a comparative fashion.
The current NOX Actor system will be used as benchmark, as that is the system we are trying to improve upon.

\ptparagraph{Faster Updates}
Faster updates can be measured in several ways, in our case we will go for a hybrid approach. 
We will record how much time that is spent over a set amount of updates. 
For example, how long time does it take to execute X number of updates, with Y sets of entities.
The testing methods will include low overhead timing of the system, as well as more in depth profiling.
This is to ensure that the time is spent mostly within the components, and not the system itself.
We will also record data in relation to cache and branch prediction.

\ptparagraph{Faster Spawning}
Faster spawning will be measured in the same ways as the faster updates, 
by doing time recordings and profiling on inclusive and exclusive function time.

\ptparagraph{Memory Usage}
Heap based memory usage will be logged with the help of a profiler. Allowing us to inspect how much memory that has been used throughout a test case.

\ptparagraph{Compile Time}
Compile time will be measured within all the test cases once they are complete.

\ptparagraph{Component Magnitude}
The component magnitude will function in the same manner as the faster updates and spawning measurements. 
