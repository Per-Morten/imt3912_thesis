\section{Quality Requirements}
Quality improvements in this spec will be split into two sections,
measurable requirements, and non measurable requirements.
The non measurable requirements can be seen more as guidelines. 
The measurable quality requirements metrics will be explained in detail in the Measuring Quality Requirements Section.

\subsection{Non Measurable Requirements}
\ptparagraph{Focus on Security}
Suttung has stated that although they want an efficient system, they still want a focus on security.
Suttung wants a system that is easy to use correctly, and hard to use incorrectly, as to decrease the chance of bugs.

\subsection{Measurable Requirements}
\ptparagraph{Faster Updates}
Updates in this specification is meant as the main update functions of the games logic.
I.e. How long it takes the whole game world to do one loop.

\ptparagraph{Faster Spawning}
Suttung has stated that they would like to see faster creation and destruction of entities within the engine, 
as this is a performance problem within their system.

\ptparagraph{Memory Usage}
Suttung has expressed that they do not want to see a large increase in memory usage, 
unless this increase can be justified through a large increase in performance.

\ptparagraph{Compile Times}
Long compile times is detrimental to the fast iteration cycles that developers strive for when developing games.
With this reasoning compile times should be kept within acceptable levels.

\ptparagraph{Component Magnitude}
The games Suttung has made so far has only contained a small number of component types. Around 30-50 components.
Suttung has requested that the new ECS can contain a large number of different component types upwards of 200, 
while still performing efficiently. 